\documentclass[11pt]{article}
% Mathematics
  % Packages
      \usepackage{amsmath, mathtools}
      \usepackage{amssymb}
      \usepackage{amsthm}
      \usepackage{tikz-cd}
  % Theorem styling
    % Number within sections
      \newcounter{number_within_section}
      %\numberwithin{number_within_section}{section}
    % Theorem-style
      \newtheorem{theorem}    [number_within_section]{Theorem}
      \newtheorem*{theorem*}  {Theorem}
      \newtheorem{corollary}  [number_within_section]{Corollary}
      \newtheorem{lemma}      [number_within_section]{Lemma}
      \newtheorem{observation}[number_within_section]{Observation}
      \newtheorem{conjecture} [number_within_section]{Conjecture}
      \newtheorem{proposition}[number_within_section]{Proposition}
    % Definition-style
      \theoremstyle{definition}
      \newtheorem{definition} [number_within_section]{Definition}
      \newtheorem{example}    [number_within_section]{Example}
  % Create \numberthis command to number part of multi-line align* displays
      \newcommand\numberthis{\addtocounter{equation}{1}\tag{\theequation}}

% Spacing and typography
	% A4 lengths
    	\newlength{\pagewidth}
    	\setlength{\pagewidth}{210mm}
    	\newlength{\pageheight}
    	\setlength{\pageheight}{297mm}

	% Set page size and margins
		% 0.15625 = 10/64
  		\usepackage[a4paper,
  			left=   0.15625 \pagewidth,
  			right=  0.15625 \pagewidth,
  			top=    0.15625 \pageheight,
  			bottom= 0.15625 \pageheight
  		]{geometry}


% Section styling
    	\usepackage{titlesec}
    	\titleformat{\section}{\normalfont\scshape\large\filcenter}{Lecture \thesection, }{0pt}{}
    	\titlespacing*{\section}{0em}{2em}{1em}

% Times new roman
      \usepackage{mathptmx}

\begin{document}

\section*{\LARGE Functional analysis}

\leftskip=1.5cm\rightskip=1.5cm

This course in functional analysis was lectured by Andr\'as Zs\'ak at the University of Cambridge during Michaelmas term 2017,
and these notes have been produced by Piet Lammers.
Please feel free to use these notes as you see fit, but do so at your own risk.

\leftskip=0cm \rightskip=0cm


\section{6 October 2017.}
% !TEX root = ../ms.tex
We recall the definitions of a \emph{vector space} and of a \emph{(linear) functional} and of a \emph{(semi)norm} on a vector space.
A \emph{normed space} is a vector space equipped with a norm.
A \emph{Banach space} is a complete normed space over either $\mathbb{R}$ or $\mathbb{C}$.
The letter $X$ will usually denote a normed space,
and we write $B_X$ and $S_X$ for the unit ball and unit sphere of $X$ respectively.
Let $X$ be a normed space over some field $k\in\{\mathbb{R},\mathbb{C}\}$
and let $f:X\to k$ be a linear functional.
We call $f$ bounded if
\begin{equation}
  \sup_{x\in B_X}|f(x)|<\infty.
  \label{dualnorm}
\end{equation}
Note that $f$ is bounded if and only if it is continuous.
The collection of all bounded linear maps $f$ is called the \emph{dual space} of $X$,
and we write
$$X^*:=\{f:X\to k:\text{$f$ is linear and continuous}\}.$$
The left hand side of (\ref{dualnorm}) norms this vector space,
and is straightforward to check that this space is a Banach space.


Let $X$ and $Y$ be two normed spaces over $k\in\{\mathbb{R},\mathbb{C}\}$.
Write $X\sim Y$ if they there exists a vector space isomorphism that is continuous
and with a continuous inverse.
Write $X\cong Y$ if they are isometrically isomorphic, meaning that the isomorphism preserves
the norm.
Finally write $\langle x,f\rangle$ for $f(x)$, so that
$\langle\cdot,\cdot\rangle$ is a bilinear map from $X\times X^*$ to $k$.
It is straightforward to check that $|\langle x,f\rangle|\leq ||x||_X||f||_{X^*}$.

Now we state the Hahn-Banach theorem.
Let $X$ be a real vector space.
A functional $p:X\to \mathbb{R}$ is called
\emph{positive homogeneous} if $p(tx)=tp(x)$ for any $t\geq 0$, $x\in X$,
and \emph{subadditive} if $p(x+y)\leq p(x)+p(y)$ for any $x,y\in X$.
The functional $p$ is called sublinear if it satisfies both these conditions.
Norms, seminorms and linear functionals are examples of sublinear functionals.

\begin{theorem}[Hahn-Banach]\label{hb1}
  Let $X$ be a real vector space and $p$ a sublinear functional.
  Let $Y$ be a subspace of $X$ and $g$ a linear functional of $Y$
  with $g\leq p$ on $Y$.
  Then $g$ extends to a linear functional $f$ of $X$ with $f\leq p$.
\end{theorem}

For the proof, we recall Zorn's lemma (which is equivalent to the axiom of choice).
Let $(P,\leq)$ be a partially ordered set.
A subset $C\subset P$ is called a \emph{chain} if $\leq$ totally orders $C$.
An element $x\in P$ is called \emph{maximal} if no element in $P$ is strictly larger than $x$
(i.e., $x\leq y$ implies $y=x$).

\begin{theorem*}[Zorn's lemma]
  If $P$ is a nonempty partially ordered set such that every chain has an upper bound,
  then $P$ contains a maximal element.
\end{theorem*}

\begin{proof}[Proof of the Hahn-Banach theorem]
  Let $P$ be the set of pairs $(h,Z)$ such that $Y\subset Z\subset X$
  and such that $h$ is a linear functional on $Z$ extending $g$ to $Z$ satisfying $h\leq p$ on $Z$.
  The set $P$ is partially ordered by extension, and nonempty as it contains
  the element $(g,Y)$.
  For a chain $\{(h_i,Z_i)\}_{i\in I}$ it is easy to construct an upper bound,
  one takes $Z=\cup_{i\in I}Z_i$ and for $h$ one takes
  $h(x)=h_i(x)$, where $i$ is such that $x\in Z_i$ ($h$ is clearly independent of this choice).
  By construction, $h$ is linear and satisfies $h\leq p$ on $Z$.
  Thus, by Zorn's lemma, $P$ contains a maximal element, say $(h,Z)$.
  To demonstrate that $Z=X$ we argue by contradiction.
  Assume that there exists an element $x\in X\setminus Z$, and define $Z_x:=Z+\mathbb{R}x$.
  Define $$h_\alpha:Z+\mathbb{R}x\to\mathbb{R},\,z+\lambda x\mapsto h(z)+\lambda\alpha.$$
  To finish the proof it suffices to demonstrates that there exists an $\alpha\in\mathbb{R}$
  such that $h_\alpha\leq p$ on $Z_x$.
  Thus we need to find $\alpha$ such that for all $\lambda\neq0$ and for any $z$,
  \begin{align*}
    h_\alpha(z+\lambda x)&\leq  p(z+\lambda x)\\
    h(z)+\lambda\alpha&\leq p(z+\lambda x)\\
    \lambda\alpha&\leq p(z+\lambda x)-h(z)\\
    \lambda\alpha&\leq p(\lambda z'+\lambda x)-h(\lambda z'),
  \end{align*}
  writing $z'=z/\lambda$ to obtain the last inequality.
  Using positive homogeneity of $p$, we only need to look at $\lambda=\pm 1$.
  So, to prove existence of $\alpha$, it suffices to show that
  $$\sup_{z\in Z} h(z)-p(z-x) \leq \inf_{z\in Z} p(z+x)-h(z).$$
  But
  $$h(z_1)-p(z_1-x)\leq p(z_2+x)-h(z_2)$$
  since
  $$h(z_1)+h(z_2)=h(z_1+z_2)\leq p(z_1+z_2)\leq p(z_1-x)+p(z_2+x).$$
  This finishes the proof of the Hahn-Banach theorem.
\end{proof}



\section{9 October 2017.}
% !TEX root = ../ms.tex
\begin{theorem}[Hahn-Banach]\label{hb2}
  Let $X$ be a vector space over $k\in\{\mathbb{R},\mathbb{C}\}$, $p$ a seminorm on $X$,
  $Y\subset X$ a subspace, and $g$ a linear functional on $Y$ satisfying
  $|g|\leq p$ on $Y$.
  Then $g$ extends to a linear functional $f$ on $X$ that satisfies $|f|\leq p$ on $X$.
\end{theorem}

\begin{proof}
Let us first look at the real case.
Note that $g\leq |g|\leq p$ on $Y$,
so that we may apply Theorem \ref{hb1} to extend $g$ to a linear functional $f$ on $X$ satisfying
$f\leq p$. But then $-f\leq p$ as well, since $-f(x)=f(-x)\leq p(-x)=p(x)$.
Hence $|f|\leq p$.

Now the complex case.
Think of $X$ as a real space and let $g_\mathbb{R}=\operatorname{Re}(g)$.
Then $|g_\mathbb{R}|\leq |g|\leq p$ on $Y$.
By the real case $g_\mathbb{R}$ extends to a $\mathbb{R}$-linear functional $f_\mathbb{R}$ of $X$ satisfying
$|f_\mathbb{R}|\leq p$ on $X$.
Define
$f$ by $f(x)=f_\mathbb{R}(x)-if_\mathbb{R}(ix)$;
it is straightforward to check that this is a $\mathbb{C}$-linear functional on $X$.
Now $\operatorname{Re}(f)|_Y=f_{\mathbb{R}}|Y=g_{\mathbb{R}}$ so $f|_Y=g$ by uniqueness.
We need to show that $|f|\leq p$.
Pick $x\in X$ and pick $\lambda\in S^1\subset \mathbb{C}$ such that $|f(x)|=\lambda f(x)=f(\lambda x)$.
Then $|f(x)|=f(\lambda x)=f_\mathbb{R}(\lambda x)\leq p(\lambda x)=p(x)$.
\end{proof}

If $X$ is a complex vector space then let $X_\mathbb{R}$ be $X$ as real vector space.
We have just proved that if $X$ is a complex normed vector space,
the map
$$(X^*)_\mathbb{R}\to(X_\mathbb{R})^*,\,f\mapsto Re(f)$$
is an isometric isomorphism.

\begin{corollary}
  Let $X$ be a normed space over $k\in\{\mathbb{R},\mathbb{C}\}$, $p$ a seminorm,
  and $x_0\in X$. Then there exists a linear functional $f$ such that $f(x_0)=p(x_0)$
  and $|f|\leq p$ on $X$.
\end{corollary}
(It is not clear to me what is the reason for including this corollary at this point)

\begin{theorem}[Hahn-Banach]\label{hb4}
  Let $X$ be a normed space over $k\in\{\mathbb{R},\mathbb{C}\}$.
  Then
  \begin{enumerate}
    \item For any $Y\subset X$ and $g\in Y^*$, $g$ extends to a linear functional
    on $X$ of equal operator norm,
    \item For any $x_0\in X$, there exists a linear functional $f$ of $X$
    of norm one that maps $x_0$ to $||x_0||$.
  \end{enumerate}
\end{theorem}

\begin{proof}
  For Statement 1, apply Theorem \ref{hb2} with $p(x)=||g||\cdot ||x||$.
  For Statement 2, apply Theorem \ref{hb2} with $p=||\cdot||$,
  and set $g:kx_0\to k,\, \lambda x_0\mapsto \lambda ||x_0||$,
  which is a linear functional of norm one (the case $x_0=0$ is trivial).
\end{proof}

The previous theorem may be viewed as a linear version of the Tietze extension theorem.
We remark that $X^*$ separates points of $X$: if $x$ and $y$ are distinct points of $X$
then there exists a bounded linear functional $f$ such that $f(x)\neq f(y)$.
The linear functional $f$ in Statement 2 of Theorem \ref{hb4} is called a \emph{norming functional}
 for $x_0$:
for $g\in B_{X^*}$,
$|\langle x_0,g\rangle|=|g(x_0)|\leq ||g||\cdot ||x_0||\leq ||x_0||$
so $||x_0||=\max\{|\langle x_0,g\rangle|:g\in B_{X^*}\}$,
and this maximum is achieved at $f$.
If $||x_0||=1$ then
the same linear functional is also called a \emph{support functional},
because $\{f=1\}$ defines a plane tangent to the unit ball in $X$ at $x_0$.


Let $X$ be a normed space. The \emph{bidual} or \emph{second dual}
of $X$ is $X^{**}=(X^*)^*$.
This is a Banach space with the operator norm.
For $\phi\in X^{**}$, $||\phi||_{X^{**}}:=\sup\{|\langle f,\phi\rangle|:f\in B_{X^*}\}$.
For $x\in X$ define $\hat x$ to be a linear functional on $X^*$ by $\hat x(f)=f(x)$.
In bracket notation, $\langle f,\hat x\rangle=\langle x,f\rangle$.
The map $\hat x$ is the evaluation map of functionals in $X^*$ at $x$.
It is clear that $\hat x$ is a linear map. Furthermore
$|\hat x(f)|=|f(x)|\leq ||f||\cdot ||x||$.
So $\hat x\in X^{**}$ and $||\hat x||\leq ||x||$.
In fact, from the existence of norming functionals (Theorem \ref{hb4}, Statement 2) it follows that
$||\hat x||= ||x||$. This proves the following theorem.

\begin{theorem}
  The map $\hat \cdot: X\to X^{**}$
  is an isometric ismorphism from $X$ to $\hat X\subset X^{**}$.
\end{theorem}
The map $\hat\cdot$ is called the \emph{canonical embedding}.
Note that $\hat X$ is closed in $X^{**}$ if and only if $X$ is complete.
Furthermore, the closure of $\hat X$ in $X^{**}$ is a Banach space,
containing an isometric copy of $X$. We have thus implicitly proved that the space $X$
has a completion.
We say that $X$ is \emph{reflexive} if $\hat X=X^{**}$, i.e.,
if the canonical embedding is surjective.
Note that it is possible $X$ is not reflexive even if
 $X$ and $X^{**}$ are isometrically isomorphic.
Examples of reflexive spaces include $L^p$ spaces (with $1<p<\infty$),
and examples of non-reflexive spaces include $c_0$, $l^1$ and $l^\infty$.











%



\section{11 October 2017.}
% !TEX root = ../ms.tex
If $X$ and $Y$ are normed spaces then we write $\mathcal{B}(X,Y)$ for the set of bounded
linear operators from $X$ to $Y$. The operator norm norms this space:
$$
  ||T||=\sup\{||Tx||:x\in B_X\}
       =\sup\{||y||:y\in T(B_X)\}
       =\sup\{||y||:y\in T(D_X)\},
$$
where $D_X:=\{x\in X : ||x||<1\}$.
If $Y$ is complete, then so is $\mathcal{B}(X,Y)$.
For $T\in\mathcal{B}(X,Y)$, its \emph{dual operator} is
$$T^*:Y^*\to X^*,\, T^*(g)=g\circ T.$$
This is well-defined and
$\langle x, T^* g\rangle = \langle Tx, g\rangle$.
It is straightforward to deduce that the operator $T^*$ is linear and bounded (i.e., $T^*\in\mathcal{B}(Y^*,X^*) $).
Moreover, $||T^*||=||T||$,
since
$$||T^*||
=\sup_{g\in B_{Y^*}} ||T^*g||
=\sup_{g\in B_{Y^*}}\sup_{x\in B_X}|\langle x,T^* g\rangle |
=\sup_{x\in B_X}\sup_{g\in B_{Y^*}}|\langle T x,g\rangle |
=\sup_{x\in B_X}||Tx||=||T||;$$
the penultimate equality follows by Theorem 4(2).

As an example, pick $1<p<\infty$, and define $R:l_p\to l_p$ to be the right shift operator.
Then its dual $R^*:l_q\to l_q$ is the left shift operator, where $p$ and $q$ are H\"older conjugates.

We now list some properties of the dual operator.
\begin{enumerate}
  \item The dual of $\operatorname{Id}_X$ is $\operatorname{Id}_{X^*}$,
  \item The operation of taking the dual of an operator is linear,
  \item For compositions of operators we have
  $(S\circ T)^*=T^*\circ S^*$,
  \item The map $\cdot^*:\mathcal{B}(X,Y)\to \mathcal{B}(Y^*,X^*),\,T\mapsto T^*$
  is an isometric linear map, but it is not surjective in general.
\end{enumerate}
The first three properties are easy to check in the bracket notation.
Now some remarks.
\begin{enumerate}
  \item If $X\sim Y$ then $X^*\sim Y^*$. For this, let $S\in \mathcal{B}(Y,X), T\in \mathcal{B}(X,Y)$
  such that $ST = \operatorname{Id}_X$ and $TS=\operatorname{Id}_Y$,
  then $T^*S^*=\operatorname{Id}_{X^*}$ and $S^*T^*=\operatorname{Id}_{Y^*}$.
  \item If $T\in \mathcal{B}(X,Y)$, then (\ref{cd1}) commutes.
  For this, one checks that $\hat {Tx}=T^{**}\hat x$ (in bracket notation).
  \item If $X$ and $Y$ are Hilbert spaces and if we identify every element with its dual,
  then the dual of an operator is its adjoint.
\end{enumerate}

\begin{equation}\label{cd1}
\begin{tikzcd}
X\arrow{r}{T}\arrow{d}{\hat\cdot} & Y \arrow{d}{\hat\cdot}\\
X^{**} \arrow{r}{T^{**}}& Y^**
\end{tikzcd}
\end{equation}


Let $X$ be a normed space,
and $Y\in X$ a closed subspace.
The \emph{quotient space} $X/Y=\{x+Y:x\in X\}$ is a normed space in the \emph{quotient norm}
$||x+Y||=\inf\{||x+y||:y\in Y\}=\operatorname{Dist}(x,Y)$.
The \emph{quotient map} $q:X\to X/Y$ is given by $q(x)=x+Y$.
The map $q$ is linear, surjective, and the norm of $q$ is one except if $Y=X$.
This last fact holds because clearly $||q||\leq 1$
and also $q(D_X)=D_{X/Y}$.
To see that this latter equality holds true,
observe that
$||x+Y||<1$ implies that there exists a $y\in Y$ such that
$x+y<1$, so $x+y\in D_X$.
Hence $q$ is an \emph{open map}
(meaning that if $U\subset X$ is open then $q(U)$ is open).

Suppose $T\in\mathcal{B}(X,Z)$ and $Y\subset \operatorname{Ker} T$, then there exists a unique map
$\tilde T:X/Y\to Z$ such that
  (\ref{cd2}) commutes.
To see this,
observe that $\tilde T$ must be defined by $\tilde T(q(x))=Tx$.
This map is linear and bounded.
To see that $\tilde T$ is bounded, observe that
$\tilde T(D_{X/Y})=\tilde T q(D_X)=T(D_X)$, so that $||\tilde T||=||T||$.

\begin{equation}\label{cd2}
\begin{tikzcd}
X\arrow{rr}{T}\arrow{rd}{q} && Z \\
& X/Y\arrow[dashed]{ur}{\tilde T} &
\end{tikzcd}
\end{equation}


\begin{theorem}
Let $X$ be a normed space. If $X^*$ is separable, then so is $X$.
\end{theorem}

Remark: The converse is false, take for example $X=l_1$. Observe that all binary sequences
are in the unit ball of $l_1^*$, and that there are overcountably many such sequences.

\begin{proof}
  Let $f_1,f_2,f_3,...$ be a dense sequence in $S_X^*$.
  Then for all $n$, there exists an $x_n\in B_X$ such that $|f_n(x_n)|>\frac{1}{2}$.
  Let $Y$ be the closure of $\operatorname{Span}\{x_n:n\in\mathbb{N}\}$.
  Then we claim that $Y=X$ (this is sufficient).
  Assume that the claim is false. Then $X/Y\neq \{0\}$, so there exists a $g\in S_{(X/Y)^*}$
  (by Theorem 4(2)). Let $f=g\circ q$ ($q$ being the quotient map).
  Then $||f||=1$ (since $f(D_X)=g(q(D_X))=g(D_{X/Y})$)
  and $Y\subset \operatorname{Ker} f$.
  For any small $\varepsilon>0$,
   there exists an $n$ such that $||f-f_n||<\varepsilon$
   (by density of $f_1,f_2,f_3,...$).
But then
$
  |f_n(x_n)|>\frac{1}{2}$ (by choice of $x_n$),
  $|(f-f_n)(x_n)|<\varepsilon$ (since $||x_n||\leq 1$)
  and $f(x_n)=0$ (since $x_n\in Y\subset \operatorname{Ker} f$).
This is a contradiction.
\end{proof}















%


\section{13 October 2017.}
% !TEX root = ../ms.tex

\begin{theorem}\label{linftyuniversal}
  Every separable normed space is isometrically isomorphic to a subspace of $l_\infty$.
\end{theorem}

\begin{proof}
  Let $X$ be separable and $(x_n)_{n\geq 0}$ a dense sequence in $X$.
  Pick, for every $n$, an element $f_n\in S_{X^*}$ such that
  $\langle x_n,f_n\rangle =1$ (Theorem 4(2)).
  Define $T:X\to l_\infty$ by $Tx=(\langle x_n,f_n\rangle )_{n\geq 1}$.
  It is well-define, since
  $|\langle x,f_n\rangle|\leq ||x||\cdot||f_n||=||x||$.
  So $T\in \mathcal{B}(X,l_\infty)$, since $||T||\leq 1$.
  Also, $||Tx_n|| \geq |\langle x_n,f_n\rangle |=||x_n||.$
  Hence $||Tx_n||=||x_n||$.
  By density and continuity,
  $||Tx||=||x||$ for any $x\in X$.
\end{proof}

Remarks.

\begin{enumerate}
  \item Theorem \ref{linftyuniversal} says that $l_\infty$ is isometrically universal for the class
  $\mathcal{SB}$ of separable Banach spaces.
  \item The dual version of Theorem \ref{linftyuniversal} says that every separable space is a
  quotient of $l_1$.
\end{enumerate}

\begin{theorem}[vector valued Liouville theorem]
Let $X$ be a complex Banach space and $f:\mathbb{C}\to X$ be analytic and bounded.
Then $f$ is constant.
\end{theorem}

Remark. Be the statement ``$f$ is analytic'' we mean that it is differentiable at any point.

\begin{proof}
Fix $\phi\in X^*$. Then $\phi\circ f:\mathbb{C}\to\mathbb{C}$ is analytic and bounded.
Hence, by the scalar Liouville theorem, $\phi\circ f$ is bounded.
Therefore, $\phi(f(z))=\phi(f(0))$ for any $z\in\mathbb{C}$.
Since $X^*$ separates points of $X$ (by Theorem 4(2)),
$f(z)=f(0)$ for any $z\in\mathbb{C}$.
\end{proof}

A \emph{locally convex space}
(LCS) is a pair $(X,\mathcal{P})$ where $X$ is a real or complex vector space,
and $\mathcal{P}$ is a family of seminorms on $X$ that separates points of $X$ in the sense
that for any nonzero element $x\in X$, there exists a seminorm $p\in\mathcal{P}$
with $p(x)\neq 0$.
Then $\mathcal{P}$ defines a topology on $X$ as follows.
A subset $U\subset X$ is open if and only if for any $x\in U$, there exists an $\varepsilon>0$
and an $n\in\mathbb{N}$ and some seminorms $p_1,...,p_n\in\mathcal{P}$
such that
$$x\in\{y\in X:\text{$p_k(y-x)<\varepsilon$ for all $1\leq k\leq n$}\}\subset U.$$

Remarks.
\begin{enumerate}
  \item Vector addition and scalar multiplication are continuous.
  \item The topology is Hausdorff since $\mathcal{P}$ separates the points of $X$.
  \item $x_n\to x$ in $X$ if and only if $p(x_n-x)\to 0$ for all $p\in\mathcal{P}$.
  \item Two families of seminorms $\mathcal{P},\mathcal{Q}$ are \emph{equivalent}
  (write $\mathcal{P}\sim\mathcal{Q}$) if they give the same topology.
  Note that $(X,\mathcal{P})$ is metrisable if and only if $\mathcal{P}$
  is equivalent to a countable family of seminorms.
\end{enumerate}

A \emph{Fr\'echet space} is a complete metrisable LCS.

Examples.

\begin{enumerate}
  \item Every normed space $(X,||\cdot ||)$ is a LCS, with $\mathcal{P}=\{||\cdot ||\}$.
  \item Let $U\subset \mathbb{C}$ be nonempty and open.
  Let $$\mathcal{O}(U)=\{f:U\to\mathbb{C}:\text{$f$ is analytic}\}.$$
  For a compact $K\subset U$, let $p_K(f)=\sup_{z\in K}|f(z)|$,
  and define $\mathcal{P}=\{p_K:\text{$K\subset U$ is compact}\}$.
  Then $(\mathcal{O}(U),\mathcal{P})$ is a LCS. The topology is the topology of local
  uniform convergence (an important topology in complex analysis).
  It is also a Fr\'echet space.
  However, the topology cannot be induced by a single norm - this is Montel's theorem.
  \item Fix $d\in\mathbb{N}$ and some nonempty open set $\Omega\subset\mathbb{R}^d$.
  For a \emph{multi-index} $\alpha\in(\mathbb{Z}_{\geq 0})^d$, we let
  $D^\alpha=(\frac{\partial}{\partial x_1})^\alpha_1\cdot ...\cdot (\frac{\partial}{\partial x_d})^\alpha_d$.
  For compact $K\subset\Omega$ and $f\in C^\infty(\Omega)$, define
  the seminorm
  $p_{K,\alpha}(f)=\sup\{|(D^\alpha f)(x)|:x\in K\}$.
  Let $\mathcal{P}$ be the family of all such seminorms.
  Then $(C^\infty(\Omega),\mathcal{P})$ is a Fr\'echet space but not normable.
\end{enumerate}

\begin{lemma}
  Let $(X,\mathcal{P})$ and $(Y,\mathcal{Q})$ be a LCS and $T:X\to Y$ a linear map.
  Then the following are equivalent:
  \begin{enumerate}
    \item $T$ is continuous,
    \item $T$ is continuous at $0$,
    \item For any $q\in\mathcal{Q}$, there exists a $C\geq 0$ and a $n\in\mathbb{N}$
    and $p_1,...,p_n\in\mathcal{P}$ such that
    $$q(Tx)\leq C\max\{p_k(x):1\leq k\leq n\}$$
    for any $x\in X$.
  \end{enumerate}
\end{lemma}

\begin{proof}
  The first two statements are equivalent since vector addition is continuous.
  We now prove that (2) implies (3). For $q\in Q$, let $V=\{y\in Y:q(y)<1\}$.
  Then $V$ is a neighbourhood of $0$ in $Y$. Since $T$ is continuous at $0$,
  there exists a neighbourhood $U$ of $0$ in $X$ such that $T(U)\subset V$.
  Without loss of generality, we may assume that
  $U\subset \{x\in X:\text{$p_k(x)\leq \varepsilon$ for $1\leq k\leq n$} \}$
  for some $\varepsilon>0,n\in\mathbb{N},p_1,...,p_n\in\mathcal{P}$.
  Let $p$ be the maximum of $p_1,...,p_n$.
  We claim that
  $q(Tx)\leq \frac{1}{\varepsilon}p(x)$ for any $x$.
  To see this, let $x\in X$.
  If $p(x)\neq 0$ then $\frac{\varepsilon x}{p(x)}\in U$, so $T(\frac{\varepsilon x}{p(x)})\in V$, ie.,
  $q(T(\frac{\varepsilon x}{p(x)}))\leq 1$, hence
  $q(Tx)\leq \frac{1}{\varepsilon}p(x)$.
  On the other hand, if $p(x)=0$, then $\lambda x\in U$ for any scalar $\lambda$,
  so $T(\lambda x)\in V$ for all $\lambda$.
  In turn $q(T(\lambda x))=|\lambda|q(Tx)<1$ for all $\lambda$,
  Therefore $q(Tx)=0$.

  We now prove that (3) implies (2).
  Let $V$ be a neighbourhood of $0$ in $Y$. We look for a neighbourhood $U$ of $0$ in $X$
  such that $T(U)\subset V$.
  Without loss of generality,
  let $V=\{y\in Y:\text{$q_k(y)<\varepsilon$ for $1\leq k\leq n$}\}$ for some
  $\varepsilon>0,n\in\mathbb{N},q_1,...,q_n\in\mathcal{Q}$.
  By (3), for each $k$, there exists a $C_k\geq 0$, $m_k\in\mathbb{N}$, $p_{k,1},...,p_{k,m_k}\in\mathcal{P}$
  such that
  $q_k(Tx)\leq C_k \max\{p_{k,j}(\lambda):1\leq j\leq m_l\}$.
  Let
  $U=\{x\in X:p_{k,j}<\varepsilon / C_k, 1\leq j\leq m_k, 1\leq k\leq n\}$
  and therefore $T(U)\subset V$.
\end{proof}













%


\section{16 October 2017.}
% !TEX root = ../ms.tex
The \emph{dual space} of a LCS $(X,\mathcal{P})$ over $k$ is
$$X^*:\{f:X\to k:\text{$f$ is linear and continuous}\}.$$

\begin{corollary}
  Let $(X,\mathcal{P})$ be a LCS over $k$ and $f:X\to k$ a linear map.
  Then $f\in X^*$ if and only if $\operatorname{Ker} f$ is closed.
\end{corollary}

\begin{proof}
  The right implication is obvious (preimages of opens are open for continuous maps).
  Now the left implication. If $f=0$ then we are done. Otherwise,
  choose $x_0\in X\setminus \operatorname{Ker}f$. Since $\operatorname{Ker}f$ is closed,
  there is a neighbourhood of $0$ such that $(x_0+U)$ is disjoint from  $\operatorname{Ker}f$.
  Without loss of generality,
  $U=\{x\in X:\text{$p_k(x\leq \varepsilon)$ for $1\leq k\leq n$}\}$
  for some $(\varepsilon,n,p_1,...,p_n)$.
  Note that $U$ is convex and \emph{balanced}
  (balanced means that $\lambda x\in U$ for any $x\in U$ and $|\lambda|=1$).
  Since $f$ is linear, $f(U)$ is also convex and balanced.
  If $f(U)=k$
  then $f(x_0+U)=f(x_0)+f(U)=k$,
  a contradiction since $(x_0+U)$ is disjoint from $\operatorname{Ker}f$.
  Otherwise, $f(U)$ is bounded.
  As in the proof of Lemma 9, we get ....
  By Lemma 9, $f$ is cts.
\end{proof}

\begin{theorem}[Hahn-Banach]
  Let $(X,\mathcal{P})$ be a LCS.
  Then
  \begin{itemize}
    \item For any $Y\subset X$, $g\in Y^*$, there is a $f\in X^*$ such that
    $g$ is the restriction of $f$ to $Y$,
    \item For any closed subspace $Y\subset X$, $x_0\in X\setminus Y$,
    there exists a $f\in X^*$ such that $f(Y)=0$ and $f(x_0)\neq 0$.
  \end{itemize}
\end{theorem}

\begin{proof}
\begin{itemize}
  \item By Lemma 9, there exist $(C,n,p_1,...,p_n)$ such that
  $|g(y)|\leq C\max p_k(y)$ for all $y\in Y$.
  Set $p(x)=C\max p_k(x)$. Then $p$ is a seminorm on $X$.
  By Theorem 2, the desired $f$ with $f\leq p$ exists.
  \item Let $Z=\operatorname{Span}\{Y,x_0\}$, define $g:Z\to k$
  by $g(y+\lambda x_0)=\lambda$.
  Then $g$ is linear, $g(Y)=0$, and $g(x_0)=1\neq 0$.
  Also, $\operatorname{Ker} g=Y$ which is closed, so by Corollary 10,
  $g\in Z^*$.
  By part 1 of this proof, $g$ extends to some $f\in X^*$.
\end{itemize}
\end{proof}

So remark again that $X^*$ separates the points of $X$.





%----------------------------

{\LARGE 2. The dual spaces of $L_p(\mu)$ and $C(K)$}

Recall that we write $(\Omega,\mathcal{F},\mu)$ for a measure space.
For $1\leq p<\infty$, we define
$$L_p(\Omega,\mathcal{F},\mu)=L_p(\mu)=\{f:\Omega\to k:\text{$f$ is measurable and $\int_\Omega |f|^pd\mu<\infty$}\}.$$
This space is normed by the $L_p$-norm,
for $f\in L_p(\mu)$ given by
$$||f||_p:=\left(\int_\Omega |f|^pd\mu\right)^{\frac{1}{p}}.$$
Then $(L_p(\mu),||\cdot ||_p)$ is a normed space provided that we identify functions that are equal almost everywhere.

We also define
$$L_\infty(\mu)=\{f:\Omega\to k:\text{$f$ is measurable and essentially bounded}\},$$
with the $L_\infty$-norm given by
$$||f||_{\infty}:=\operatorname{Ess Sup}|f|.$$
Then again $(L_\infty(\mu),||\cdot ||_\infty)$ is a normed space provided that we identify functions that are equal almost everywhere.

Remark. If $X$ is a real or complex vector space and $||\cdot ||$ a seminorm on $X$,
then $N=p^{-1}(0)$ is a subspace of $X$ and $X/N$ is a normed subspace with the norm provided by
$||x+N||=||x||$.
We will not do this for $L_p(\mu)$.

\begin{theorem}
The space $L_p(\mu)$ is complete for $1\leq p\leq \infty$.
\end{theorem}

\begin{proof}
First, consider $p\leq \infty$.
Let $f_n$ be a sequence in $L_p(\mu)$ such that $\sum ||f_n||_p<\infty$.
Let $g_n=\sum_{k=1}^n f_k$
and $s_n=\sum_{k=1}^n |f_k|$
and $s=\lim_{n}s_n$ (which takes values in $[0,\infty]$).
Let $M=\sum_{n=1}^\infty||f_n||_p$.
Let $A=\{\omega\in\Omega:s(\omega)=\infty\}$.
By the triangular inequality,
$$||s_n||_p\leq \sum_{k=1}^n||f_k||_M\leq M$$
for all $n$, so $\int s_n^p\leq M^p$for all $n$.
Since $s_n$ is increasing and this bound is uniform over $n$, we must have $\mu(A)=0$.
Therefore define
$f(\omega)=\sum_{k=1}^\infty f_k(\omega)$ for $\omega \not\in A$,
and $f=0$ on $A$.
Now show that $g_n\to f$ in $L_p$-norm.

Now the case $p=\infty$; assume that $f_n$ is Cauchy in $L_\infty(\mu)$.
For every $i$ choose a null-set $N_i$ such that
$$||f_i||_\infty=\sup_{\Omega\setminus N_i}|f_i|$$
and for every pair $(i,j)$ choose a null-set $N_{i,j}$ such that
$$||f_i-f_j||_\infty=\sup_{\Omega\setminus N_{i,j}}|f_i-f_j|.$$
Define $N$ to be the union of all those sets $N_i$ and $N_{i,j}$,
which is again a null-set.
Then the restrictions of $f_n$ to $\Omega\setminus N$
is a uniformly converging Cauchy sequence of bounded functions,
hence converges uniformly on $\Omega\setminus N$ to some $f$.
Extend $f$ to $\Omega$ by setting $f=0$ on $N$.
\end{proof}

Let $\Omega$ be a set $\mathcal{F}$ a $\sigma$-algebra on $\Omega$.
A \emph{complex measure} on $\mathcal{F}$
is a countably additive set function
$\nu:\mathcal{F}\to\mathbb{C}$.
The \emph{total variation measure} of $\nu$ is $|\nu|:\mathcal{F}\to [0,\infty]$
$$|\nu|(A)=\sup\left\{\sum_{k=1}^n |\nu(A_k)|:\text{$\{A_1,...,A_n\}$ is a measurable partition of $A$}\right\}.$$
The measure $|\nu|$ is a positive measure.














  %


\section{18 October 2017.}
% !TEX root = ../ms.tex

As in last lecture, let $\nu$ be a complex measure.
Recall that $|\nu|$ is a positive measure that satisfies
$$|\nu(A)|\leq |\nu|(A)$$
for any measurable $A$, and that $|\nu|$ is the smallest measure satisfying that inequality.
If $A_n$ is an increasing sequence of measurable sets then
$\nu(\cup_n A_n)=\lim_{n\to\infty}\nu(A_n)$
and if $A_n$ is a decreasing sequence of measurable sets then
$\nu(\cap_n A_n)=\lim_{n\to\infty}\nu(A_n)$.
A \emph{signed measure} on $\mathcal{F}$ is a countably additive set function
$\nu:\mathcal{F}\to\mathbb{R}$ (i.e., it is a complex measure that takes real values on all measurable sets).

\begin{theorem}\label{decomp}
  Let $(\Omega,\mathcal{F})$ be a measurable space and $\nu$ a signed measure.
  Then there exists a measurable partition $\Omega=P\cup N$ such that
  $\nu$ takes nonnegative values on measurable subsets of $P$ and nonpositive values on
  measurable subsets of $N$.
\end{theorem}

Remarks.
\begin{enumerate}
  \item The partition from the previous theorem is called the \emph{Hahn decomposition of $\nu$}.
  \item Define $\nu^+(A):=\nu(A\cap P)$ and $\nu^-(A):=-\nu(A\cap N)$ for any measurable $A$.
  Then $\nu^+$ and $\nu^-$ are finite positive measures and $\nu=\nu^+-\nu^-$,
  and $|\nu|=\nu^++\nu^-$.
  Moreover, these two properties uniquely determine $\nu^+$ and $\nu^-$
  (since $\nu^+=\frac{1}{2}(\nu+|\nu|)$ and $\nu^-=\frac{1}{2}(|\nu|-\nu)$).
  This is called the \emph{Jordan decomposition} of $\nu$.
  \item If $\nu$ is a complex measure, then $\operatorname{Re}(\nu)$ and $\operatorname{Im}(\nu)$
  are signed measures, having Jordan decompositions $\nu_1-\nu_2$ and $\nu_3-\nu_4$ respectively.
  Then $\nu=\nu_1-\nu_2+i\nu_3-i\nu_4$ is called the \emph{Jordan decomposition} of the complex measure $\nu$.
  Moreover, $\nu_k\leq |\nu|$ for any $k$,
  and $|\nu|\leq \nu_1+\nu_2+\nu_3+\nu_4$,
  so $|\nu|$ must be finite.
  \item In Remark 2, $\nu^+(A)=\sup\{\nu(B):B\subset A,B\in\mathcal{F}\}$.
  Then $\nu(B)=\nu^+(B)-\nu^-(B)\leq \nu^+(B)\leq \nu^+(A)$.
  Thus $B=P\cap A$ and $\nu(B)=\nu^+(A)$.
\end{enumerate}

\begin{proof}[Proof of Theorem \ref{decomp} (non-examinable)]
Define $\nu^+(A)=\sup\{\nu(B):B\subset A,B\in\mathcal{F}\}$.

Since $\varnothing\subset A$, we have $\nu^+:\mathcal{F}\to[0,\infty]$

and if $A$ and $B$ are disjoint then $\nu^+(A\cup B)=\nu^+(A)+\nu^+(B)$.
\begin{enumerate}
  \item Claim: $\nu^+(\Omega)<\infty$.
  Assume $\nu^+(\Omega)=\infty$. Then there exists a set $A_1\subset \Omega$ such that $\nu(A_1)>1$.
  Then $\nu^+(\Omega)=\nu^+(A_1)+\nu^+(\Omega\setminus A_1)=\infty$.
  One of these two sets must have $\nu^+$-measure $\infty$, say $B_1$.
  Then from $B_1$ select a subset $A_2$ having $\nu$-measure at least $2$.
  Let $B_2$ equal $A_2$ or $B_1\setminus A_2$, whichever has $\nu^+$-measure $\infty$.
  Continuing like this, we get a decreasing sequence of sets $B_n$
  such that $\nu^+(B_n)=\infty$ for all $N$
  and $A_n\subset B_{n-1}$ and $\nu(A_n)>n$.
  Then either there exist infinitely many indices $k_n$ such that $B_{k_n}=B_{k_n-1}\setminus A_{k_n}$,
  so that the $A_{k_n}$ are pairwise disjoint,
  a contradiction since it implies $\nu(\cup_n A_{k_n})=\infty$.
  Otherwise, the $A_n$ form, for $n>N$ for some $N$,
  a decreasing sequence of sets.
  Hence $\nu(\cap_{n\geq N}A_n)=\lim_{n} \nu(A_n)=\infty$,
  a contradiction.
  \item Claim: there exists a $P\subset\Omega$ such that $\nu^+(\Omega)=\nu(P)$.
  If $\nu(B)>\nu^+(\Omega)-\varepsilon$,
  and $\nu(C)>\nu^+(\Omega)-\delta$
\end{enumerate}
\end{proof}

Let $(\Omega,\mathcal{F},\mu)$ be a measure space. Let $\nu$ be a complex measure on the same measurable space.
We say that $\nu$ is \emph{absolutely continuous} with respect to $\mu$ if for any measurable set $A$,
we have $\mu(A)=0$ implies that $\nu(A)=0$.
We write $\nu<<\mu$.

Remarks.
\begin{enumerate}
  \item If $\nu<<\mu$ then $|\nu|<<\mu$.
  If $\nu=\nu_1-\nu_2+i\nu_3-i\nu_4$ is the Jordan decomposition of $\nu$,
  then $\nu_k<<\mu$ for all $k$.
  \item We have $\nu<<\mu$ if and only if for all
  for all $\varepsilon>0$, there exists a $\delta>0$,
  such that for any measurable set $A$, $\mu(A)<\delta$ implies $|\nu(A)|<\varepsilon$.
\end{enumerate}

Example. If $f\in L_1(\mu)$, and $\nu(A):=\int_A fd\mu$, then $\nu$ is a complex measure that is
absolutely continuous with respect to $\mu$.

Let $(\Omega, \mathcal{F},\mu)$ be a measure space. Say a measurable set $A$ is $\sigma$-finite
if there exists a measurable increasing sequence $A_n$ such that $A=\cup_n A_n$ and $\mu(A_n)<\infty$
for all $n$.
Say $\mu$ is $\sigma$-finite if $\Omega$ is $\sigma$-finite.

\begin{theorem}[Radon-Nikodym] Let $(\Omega, \mathcal{F},\mu)$ be a $\sigma$-finite measure space,
  $\nu$ a complex measure on the same measurable space, absolutely continuous with respect to $\mu$.
  Then there exists a unique $f\in L_1(\mu)$ such that
  $\nu (A)=\int_A fd\mu$.
\end{theorem}

\begin{proof}[Proof (non-examinable)]
Uniqueness is obvious, so we will focus on existence.
Without loss of generality, $\nu$ is a finite positive measure,
and $\mu$ is finite.
Define $$\mathcal{H}=\{f:\Omega\to [0,\infty):\forall A,\,\int_A fd\mu \leq \nu(A) \}.$$
Then $0\in\mathcal{H}$.
Define $$\alpha=\sup_{f\in\mathcal{H}}\int_\Omega f d\mu \leq \nu(\Omega).$$
Now pick an increasing sequence $f_n$ in $\mathcal{H}$
such that $\int_\Omega f_n d\mu > \alpha-\frac{1}{n}$.
Then $g_n=f_1\wedge $
\end{proof}























%


\section{20 October 2017.}
% !TEX root = ../ms.tex


%%
Duals of $L_p$.

Let $(\Omega,\mathcal{F},\mu)$ be a measure space
and $L_p$ the corresponding $L_p$ space for $1\leq p\leq\infty$.

Fix $1\leq p<\infty$ and pick $1<q\leq \infty$ such that they are H\"older conjugates.
For $g\in L_q$ define $\phi_g:L_p\to k$ by $\phi_g(f)=\int_\Omega fgd\mu$.
By H\"olders inquality, this linear functional is well defined and
$\phi_g(f)\leq ||f||_p||g||_q$.
Therefore it is continuous: $\phi_g\in L_p^*$
and $||\phi_g||\leq ||g||_q$.
Hence $\phi:L_q\to L_p^*$ is linear and bounded (since $||\phi||\leq 1$).

\begin{theorem}
Let $(\Omega,\mathcal{F},\mu)$ and $p$, $q$ and $\phi$ be as above.
Then
\begin{enumerate}
  \item If $1<p<\inft$, then $\phi$ is an isometric isomorphism, i.e., $L_p^* \tilde = L_q$,
  \item If $p=1$ and $\mu $ is $\sigma$-finite, then $\phi$ is an isometric isomorphism,
  i.e., $L_1^* \tilde = L_\infty$.
\end{enumerate}
\end{theorem}

\begin{proof}
  We first prove that $\phi$ is isometric.
  Fix $g\in L_q$.
  Then we need that $||\phi_g||\geq ||g||_q$.
  Let $\lambda:\Omega\to k$ measureable such that $|\lambda =1|$ almost everywhere
  and such that $\lambda g=|g|$ almost everywhere.
  In (1), let $f=\lambda |g|^{q-1}$.
  Then $\int |f|^p=\int |g|^{pq-p}=\int |g|^q=||g||_q^q\leq \infty$.
  Hence $f\in L_p$ and $||f||_p=||g||_q^{q/2}$.
  Then $$||\phi_g||\cdot ||f||_p\geq |\phi_g(f)|=|\int f g | = \int |g|^q=||g||_q^q=||g||_q\cdot ||f||_p.$$
  For (2), let us assume that $g\neq 0$, so that $||g||_\infty>0$.
  Fix $0<s<||g||_\infty$.
  Let $A=\{g>s\}\subset \Omega$, which has nonzero measure.
  Since $\mu$ and hence $A$ are $\sigma$-finite,
  we can pick a subset $B$ of $A$ of positive (in particular finite) measure.
  Thus $\lambda 1_B\in L_1$.
  Hence
  $$||\phi_g||\cdot \mu(B)=||\phi_g||\cdot ||\lambda 1_B||_1\geq |\phi_g (\lambda 1_B)|=\int_B |g|\geq s\mu (B)$$
  Since $\mu(B)>0$, $||\phi_g||\geq s$.
  Hence $||\phi_g||\geq ||g||_\infty$.

  Now we are going to prove that $\phi$ is surjective.
  Fix $\psi\in L_p^*$.
  So we are looking for a $g\in L_q$ such that $\phi_g$.
  First, let $\mu$ be finite. For $f\in L_\infty$,
  $$||f||_p=\left(\int_\Omega f^p\right)^{1/p}\leq \mu(\Omega)^{1/p}||f||_\infty<\infty.$$
  Hence $L_\infty \subset L_p$. In particular, $1_A\in L_p$ for any measurable set $A$.
  Define $\nu(A)=\psi(1_A)$ for such $A$.
  If $A$ is measurable and $(A_n)_{n\geq 0}$ is a measurable partition,
  then $\sum_{k=1}^n 1_{A_k}\to 1_A$ almost everywhere
  and
  $$1_A-\sum_{k=1}^n 1_{A_k}\leq 1_A\in L_1.$$
  By dominated convergence, the partial sum converges to the indicator in $||\cdot ||_p$.
  Since $\psi\in L_p^*$,
  $\sum_{k=1}^n\nu (A_k)=\psi(\sum_{k=1}^n 1_{A_k})\to \psi(1_A)=\nu(A)$.
  So $\nu$ is a complex measure absolutely continuous with respect to $\mu$;
  if $\mu(A)>0$, then $1_A=0$ almost everywhere, so $\psi(1_A)=0$.
  By Radon-Nikodym, there exists a $g\in L_1$ such that $\psi(1_A)=\int_A gd\mu$ for any measurable $A$.
  By linearity,
  $\psi(f)=\int fgd\mu$ for any simple function $f$.
  The map $f\mapsto \int fgd\mu$ is a bounded linear map on $L_\infty$ (H\"older).
  Given $f\in L_\infty$, there exists a sequence of simple functions converging to $f$ in $L_\infty$.
  Hence we have $\psi(f)=\int fg d\mu$ for any $f\in L_\infty$.

  Claim: $g\in L_1$ (note that this is sufficient since then $\psi, \phi_g\in L_p^*$, $\psi=\psi_g$ on $L_\infty$,
  and $L_\infty$ is dense in $L_p$ with respect to $||\cdot ||_p$).

  Proof of claim.
  Fix $\lambda$ such that $\lambda g=|g|$ almost everywhere.
  If $1<p<\infty$ let $A_n =\{ |g|\leq n\}\subset \Omega$.
  Then $f_n=\lambda |g|^{q-1}1_{A_n}\in L_\infty$,
  so that
  $$\int f_n g=\int |g|^{1} 1_{A_n}=\psi(f_n)\leq ||\psi||\cdot ||f_n||_p =||\psi||\left(\int |g|^q 1_{A_n}\right)^{1/p}.$$
  Hence $$\int |g|^q 1_A \leq ||\psi||^q.$$
  By monotone convergence, $||g||_q<\infty$.

  If $p=1$ and $q=\infty$.

  Fix $s>||\psi ||$.
  Let $A=\{|g|>s\}\subset \Omega$.
  Let $f=\lambda 1_A$.
  Then $$\int fg= \psi(f)\leq ||\psi|| \cdot ||f||_1\leq ||\psi||\mu(A)$$
  and $s\mu(A)\leq \int_A|g|$,
  hence $\mu(A)=0$.
  Hence $||g||_\infty\leq ||\psi||<\infty$.

  The rest of the proof is non-examinable.

  For measurable $A$, let $\mathcal{F}_A$ be the set of measurable subsets of $A$,
  and $\mu_A$ the restriction of $\mu$ to $(A,\mathcal{F}_A)$.
  Then $L_p(\mu_A)$ is isometrically embedded in $L_p(\mu)$.
  Let $\psi_A$ be the restriction of $\psi$ to $L_p(\mu_A)$.
  If $A$ and $B$ are disjoint and measurable,
  then
  $$\psi_{A\cup B}=\begin{cases}
  (||\psi_A||^q+||\psi_B||^q)^{1/q} &\text{ for $1<p<\infty$}\\
  \max \{||\psi_A||,||\psi_B||\}& \text{ if $p=1$}
  \end{cases}.$$
  For $(a,b)\in\mathbb{R}^2$, $||(a,b)||_q=\sup\{ar+bt:||(r,t)||_p\leq 1\}$.

  Case 2. $\mu$ is $\sigma$-finite.

  Case 3. General $\mu$. ($1<p,q<\infty$).
  $||\psi||=\lim |\psi(f_n)|$
  $\{f_n\neq 0\}$ is $\sigma$-finite
  so there exists a sigma finite $A$ such that $||\psi||=||\psi_A||$
  and apply case $2$.
  Then $g:A\to k$ set $g=0$ on the complement of $A$.
  If $A$ and $B$ are disjoint then
  $$||\psi_{A\cup B}||=(||\psi_A||^q+||\psi_B||^q)^{1/q}$$
  and therefore $\psi_B=0$.
\end{proof}

\begin{corollary}
  If $(\Omega,\mathcal{F},\mu)$ is a measure space and $1<p<\infty$
  then $L_p(\mu)$ is reflexive.
\end{corollary}

\begin{proof}
Let $q$ be the H\"older conjugate of $p$.
Let $\phi: L_q\to L_p^*$
and $\psi: L_p\to L_q^*$
be as in the previous theorem.
Then $\phi^*:L_p^{**}\to L_q^*$ is also an isometic ismorphism.
We need to show that
$$(\phi^*)^{-1}\psi(f)=\hat f$$
for all $f$.
Given $f\in L_p$, we need that $\psi(f)=\phi^*(\hat f)$. Given $g\in L_q$,
$$\langle g,\phi^*(\hat f)\rangle =\langle \phi(g),\hat f\rangle =\langle f,\phi(g)\rangle =\int fg =\langle g,\psi (f)\rangle.$$
\end{proof}




















%


\section{23 October 2017.}
% !TEX root = ../ms.tex

Dual of $C(K)$.

Here $K$ will denote a compact Hausdorff space.
We will write
\begin{align*}
  C(K)&:=\{f:K\to\mathbb{C}:\text{$f$ is continuous}\},\\
  C^\mathbb{R}(K)&:=\{f:K\to\mathbb{R}:\text{$f$ is continuous}\},\\
  C^+(K)&:=\{f\in C(K):\text{$f\geq 0$ on $K$}\},\\
  M(K)&:=\{\phi: C(K)\to\mathbb{C}:\text{$\phi$ is linear and continuous}\},\\
  M^\mathbb{R}(K)&:=\{\phi\in M(K):\text{$\phi(f)\in \mathbb{R}$ if $f\in C^\mathbb{R}(K)$}\},\\
  M^+(K)&:=\{\phi: C(K)\to\mathbb{C}:\text{$\phi(f)$ is linear and $\phi(f)\geq 0$ if $f\in C^+(K)$}\}.
\end{align*}

Remarks.
\begin{enumerate}
  \item $C(K)$ is a complex Banach space in the uniform norm $||f||_\infty=\sup_K |f|$.
  \item $C^\mathbb{R}(K)$ is a real Banach space in the uniform norm $||f||_\infty=\sup_K |f|$.
  \item $C^+(K)$ is a subset of $C^\mathbb{R}(K)$.
  \item $M(K)=C(K)^*$ is a complex Banach space in the operator norm.
  \item $M^\mathbb{R}(K)$ is a closed real-linear subspace of $M(K)$.
  We shall soon show that  $M^\mathbb{R}(K)\sim = C^\mathbb{R}(K)^*$.
  \item Elements in $M^+(K)$ are called \emph{positive linear functionals}.
  We shall soon show that $M^+(K)\subset M(K)$.
\end{enumerate}

\begin{lemma}
\begin{enumerate}
  \item For $\phi\in M(K)$ there exist unique $\phi_1,\phi_2\in M^{\mathbb{R}}(K)$
  such that $\phi = \phi_1+i\phi_2$.
  \item The map $\phi\mapsto \phi|_{C^\mathbb{R}(K)}:M^\mathbb{R}\to C^\mathbb{R}(K)^*$
  is an isometric, real-linear isomorphism.
  \item $M^+(K)\subset M(K)$ and moreover,
  $M^+(K)=\{\phi\in M(K):||\phi||=\phi(1_K)\}$.
  \item For $\phi\in M^\mathbb{R}(K)$ there exist unique $\phi^+,\phi^-\in M^+(K)$
  such that $\phi=\phi^+-\phi^-$ and $||\phi||=||\phi^+||+||\phi^-||$.
\end{enumerate}
\end{lemma}

\begin{proof}
  \begin{enumerate}
    \item For $\phi\in M(K)$ define $\bar\phi(f)=\bar{\phi(\bar f)}$
    for any $f\in C(K)$.
    Then $\bar \phi\in M(K)$
    and $\bar {\lambda \phi+\mu\psi}=\bar\lambda\bar\phi+\bar\mu\bar\psi$
    and $\bar{\bar \phi}=\phi$.
    Also,
    $\phi=\bar\phi$ if and only if $\phi\in M^\mathbb{R}(K)$.

    Uniqueness. Assume that $\phi=\phi_1+i\phi_2$ as in the statement.
    Then $\bar\phi=\phi_1-i\phi_2$,
    so $\phi_1=\frac{1}{2}(\phi+\bar\phi)$
    and $\phi_2=\frac{1}{2i}(\phi-\bar\phi)$.
    This also proves existence.

    \item For $\phi\in M^\mathbb{R}(K)$, $\phi_1=\phi|_{C^\mathbb{R}(K)}\in C^\mathbb{R}(K)^*$
    and $||\phi_1||\leq ||\phi||$.
    Given $f\in C(K)$, choose $\lambda\in \mathbb{C}$, $|\lambda|=1$ such that
    $|\phi(f)|=\lambda\phi(f)$.
    Then
    $$|\phi(f)|=\phi(\lambda f)=
    \phi(\operatorname{Re}(\lambda f))+i\phi(\operatorname{Im}(\lambda f))
    =\phi(\operatorname{Re}(\lambda f))=\phi_1(\operatorname{Re}(\lambda f))
    \leq ||\phi_1|| \cdot ||\operatorname{Re} (\lambda f)||_\infty
    \leq ||\phi_1|| \cdot ||f||_\infty.
    $$
    Hence $||\phi||\leq||\phi_1||$.
    The map $\phi\mapsto \phi|_{C^\mathbb{R}(K)}$ is isometric.
    We now check if it is onto.
    Given $\psi\in C^\mathbb{R}(K)^*$,
    let $\phi(f)=\psi(\operatorname{Re} f)+i\psi (\operatorname{Im}f)$.
    Then $\phi\iun M(K)$ and $\phi|_{C^\mathbb{R}(K)}=\psi$.

    \item Let $\phi\in M^+(K)$. Given $f\in C^\mathbb{R}(K)$,
    we write $f=f_1-f_2$ with $f_1,f_2\in C^+(K)$.
    Then $\phi(f)=\phi(f_1)-\phi(f_2)\in\mathbb{R}$.
    Let $f\in C^\mathbb{R}(K)$ with $||f||_\infty\leq 1$.
    Then $-1\leq f\leq 1$ on $K$.
    So $-\phi(1_K)\leq \phi(f)\leq \phi(1_K)$
    and $|\phi(f)|\leq \phi(1_K)$.
    Hence we must have equality.

    Now assume that $\phi\in M(K)$ and $||\phi||=\phi(1_K)$.
    Without loss of generality, we assume that $\phi(1_K)=1$.
    Given $f\in C^\mathbb{R}(K)$, let $\phi(f)=\alpha+i\beta$ for some real $\alpha$ and $\beta$.
    We want that $\beta=0$.
    For $t\in\mathbb{R}$, let $g_t=f+it1_K$.
    Then $|\phi(g_t)|^2=|\alpha+i(\beta+t)|^2=\alpha^2+\beta^2+2\beta t+t^2\leq ||g_t||_\infty^2=||f||_\infty^2+t^2$.
    But $2\beta t\leq ||f||_\infty^2-\alpha^2-\beta^2$ for all $t$, so $\beta=0$
    and $\phi\in M^\mathbb{R}(K)$.

    Given $f\in C^+(K)$, we need that $\phi(f)\geq 0$. Without loss of generality,
    $0\leq f\leq 1_K$.
    So $0\leq 1_K-f\leq 1_K$.
    Hence $\phi(1_K-f)\leq |\phi(1_K-f)|\leq ||1_K-f||_\infty\leq 1$,
    so $\phi(f)\geq 0$.

    \item For $f\in C^+(K)$ define $\phi^+(f)=\sup\{\phi(g):0\leq g\leq f\}$.
    Since $0\leq f$, if $g\leq f$ then $||g||_\infty\leq ||f||_\infty$,
    it follows that $0\leq \phi^+(f)\leq ||\phi||\cdot ||f_||_\infty$.
    For $t\in [0,\infty)$, $\phi^+(tf)=t\phi^+(f)$.
    Given $f_1,f_2\in C^+(K)$, we claim that $\phi^+(f_1+f_2)=\phi^+(f_1) +\phi^+(f_2)$.
    Given $0\leq g_1\leq f_1$ and $0\leq g_2\leq f_2$,
    we have $g_1+g_2\leq f_1+f_2$,
    so $\phi^+(f_1+f_2)\geq \phi (g_1+g_2)=\phi(g_1)+\phi(g_2)$.
    Taking the supremum over $g_1$ and $g_2$,
    we get
    $\phi^+(f_1+f_2)\geq \phi^+(f_1)+\phi^+(f_2)$.
    Given $0\leq g\leq f_1+f_2$,
    let $g_1=g\wedge f_1$ and $g_2=g-g_1$.
    Then $0\leq g_1\leq f_1$ and $0\leq g_2\leq f_2$.
    So $\phi(g)=\phi(g_1)+\phi(g_2)\leq \phi^+(f_1)+\phi^+(f_2)$.
    Taking the supremum over $g$ gives us that
    $\phi^+(f_1+f_2)\leq \phi^+(f_1)+\phi^+(f_2)$.
    Define $\phi^+$ on $C^\mathbb{R}(K):\phi^+(f)=\phi^+(f_1)-\phi^+(f_2)$
    where $f=f_1-f_2$, $f_1,f_2\geq 0$.
    This is well-defined and real-linear.
    Define $\phi^+$ in $C(K)$
    by $\phi^+(f)=\phi^+(\operatorname{Re} f)+i\phi^+(\operatorname{Im} f)$;
    a complex-linear function.
    By definition, $\phi^+\in M^+(K)$,
    so by (3), $\phi^+\in M(K)$, $||\phi^+||=\phi^+(1_K)$.
    Define $\phi^-=\phi^+-\phi$.
    For $f\in C^+(K)$,
    $0\leq f\leq f$,
    so $\phi^+(f)\geq \phi(f)$, and therefore $\phi^-(f)\geq 0$.

    So $\phi^-\in M^+(K)$ and $\phi=\phi^+-\phi^-$.
    Hence $||\phi||\leq ||\phi^+||+||\phi^-||$.
    Given $0\leq g\leq 1_K$,
    $-1_K\leq 2g-1_K\leq 1_K$,
    so $||2g-1_K||\leq 1$.
    So $2\phi(g)-\phi(1_K)\leq ||\phi||$.
    Taking the supremum over $g$:
    $2\phi^+(1_K)-\phi(1_K)\leq ||\phi||$
and
    $||\phi^+||+||\phi^-|| =\text{ (by (3)) }\phi^+(1_K)+\phi^-(1_K)\leq ||\phi||$.

    Uniqueness. Suppose that $\phi=\phi^+-\phi^-$
    and $||\phi||=||\psi^+||+||\psi^-||$
    for $\psi^+,\psi^-\in M^+(K)$.
    For $0\leq g\leq f$, we have $\phi(g)\leq\psi^+(g)\leq \psi^+(f)$,
    so $\phi^+(f)\leq \psi^+(f)$.
     $\phi^-=\phi^+-\phi\leq \psi^+-\phi=\psi^-$.
     Thus $\psi^+-\phi^+$ and $\psi^--\phi^-$ are in $M^+(K)$.
     By (3), $||\psi^+-\phi^+||+||\psi^--\phi^-||=\psi^+(1_K)-\phi^+(1_K)+\psi^-(1_K)+\phi^-(1_K)=||\phi||-||\phi||=0$
  \end{enumerate}
\end{proof}




























%


\section{25 October 2017.}
% !TEX root = ../ms.tex
Aim is to descrive $M(K)=C(K)^*$
and $M^\mathbb{R}(K)\sim = C^\mathbb{R}(K)^*$.
Lemma 6 tells us that it is enough to consider
$M^+(K)$.

\emph{Topological preliminaries}

\begin{enumerate}
  \item As $K$ is compact and Hausdorff, it must also be normal. If $E,F\subset K$
  are disjoint and closed, then there exist
  disjoint open sets $U,V$ containing $E$ and $F$ respectively.
  This is equivalent to the following:
  If $E\subset U\subset K$ with $E$ closed and $U$ open,
  then there exists an open set $V$ such that
  $$E\subset V\subset \bar V\subset U$$
  \item
  Urysohn's Lemma.
  If $E$ and $F$ are disjoint closed subsets of $K$,
  then there exists a continuous function $f:K\to [0,1]$
  such that $f=0$ on $E$ and $f=1$ on $F$.
  \item Notation. $f\prec U$ means that $U$ is a open in $K$ and that $f:K\to[0,1]$
  is a continuous function with its support contained in $U$.
  $E\prec f$ means that $E$ is closed in $K$,
  and that $f:K\to[0,1]$
  is a continuous function with $E$ contained in $\{f=1\}$.
  \item Urysohn's Lemma is equivalent to:
  given $E\subset U\subset K$, $E$ closed, $U$ open,
  there exists an $f$ such that
  $E\prec f\prec U$.
\end{enumerate}

\begin{lemma}
  Assume that $E\subset \cup_{i=1}^n U_i$,
  where $E$ is closed, and the $U_i$ open. Then
  \begin{enumerate}
    \item there exist open $V_i$ such that $E\subset \cup_{i=1}^{n} V_i$
  and $\bar V_i \subset U_i$
  \item There exist $h_i\prec U_i$ such that $0\leq \sum_{i=1}^n h_i\leq 1$
  on $K$ and such that $\sum_{i=1}^n h_i=1$ on $E$.
  \end{enumerate}
\end{lemma}

\begin{proof}
  \begin{enumerate}
    \item By induction on $n$.
    The case $n=1$ is easy.
    Now $n>1$. $E\setminus U_n\subset \cup_{i=1}^{n-1} U_i$. By induction there exist open sets $V_i$
    such that $\bar V_i\subset U_i$ and $E\setminus U_n\subset \cup_{i=1}^{n-1}V_i$.
    Then $E\setminus\cup_{i=1}^{n-1}V_i\subset U_n $.
    Now apply the same as for $n=1$.
    \item Apply $i$ to get open sets $V_i$.
    By Urysohn's Lemma, there exist functions $g_i$ such that
    $\bar V_i \prec g_i\prec U_i$.
    Note that
    $K\setminus \cup_{i=1}^n V_i \prec g_{n+1}\prec K\setminus E$.
    Let $g=\sum_{i=1}^{n+1}$
    and $h_i=g_i/g$ for $i\leq n$. Note that $g\geq 1$ on $K$.
    So on $K$, $$0\leq \sum_{i=1}^n h_i\leq \sum_{i=1}^{n+1}g_i/g=1$$
    and on $E$,
    $g_{n+1}=0$ so that the $h_i$ sum to $1$.
  \end{enumerate}
\end{proof}


Let $X$ be a topological space.
Write $\mathcal{G}$ for the topology of $X$
and write $\mathcal{B}$ for $\sigma(\mathcal{G})$.
Members of $\mathcal{B}$ are called \emph{Borel sets}.
A \emph{Borel measure} on $X$ is measure on $(X,\mathcal{B})$.
A positive Borel measure $\mu$ on $X$ is \emph{regular} if
\begin{enumerate}
  \item $\mu(E)<\infty$ for any compact subset $E$ of $X$.
  \item $\mu(A)=\inf \{\mu (U):A\subset U\in \mathcal{G}\}$.
  \item $\mu(A)=\sup \{\mu (E):\text{$E\subset A$, $E$ is compact}\}$.
\end{enumerate}
A complex Borel measure $\nu$ on $X$ is \emph{regular}
 if $|\nu| $ is regular.
 If $X$ is compact and Hausdorff and $\mu$ a positive Borel measure on $X$ then the following
 are equivalent:
 \begin{enumerate}
   \item $\mu$ is regular,
   \item $\mu(X)<\infty$ and $\mu(A)=\inf \{\mu (U):A\subset U\in \mathcal{G}\}$,
   \item $\mu(X)<\infty$ and $\mu(A)=\sup \{\mu (E):\text{$E\subset A$, $E$ is compact}\}$.
 \end{enumerate}
 Let $\Omega$ be a set, $\mathcal{F}$ a $\sigma$-algebra on $\Omega$, $\nu:\mathcal{F}\to\mathbb{C}$
 a complex measure.
 A measurable $f:\Omega\to\mathbb{C}$ is $\nu$-integrable if $\int_\Omega|f|d|\nu|<\infty$.
 Define
 $$\int_\Omega f\d\nu=\int_\Omega f d\nu_1-\int_\Omega f d\nu_2+i\int_\Omega f d\nu_3-i\int_\Omega f d\nu_4$$
 where $\nu_i$ is the Jordan decomposition.

 Recall that
 $$\nu_i\leq |\nu|\leq \sum_i\nu_i$$
 so $f$ is $\nu$-integrable if and only if $f$ is $\nu_i$-integrable for all $i$.

 Properties
\begin{enumerate}
  \item $\int_\Omega 1_A d\nu=\nu(A)$
  \item Linearity.
  If $f$ and $g$ are $\nu$-integrable and $a,b\in \mathbb{C}$,
  then $af+bg$ is $\nu$-\integrable and
  $$\int(af+bg)d\nu=a\int fd\nu+b\int gd\nu.$$
  \item Dominated convergence theorem.
  If $f_n,f:\Omega\to \mathbb{C}$ are measurable,
  $f_n\to f$ almost everywhere,
  all dominated by $g\in L_1(|\nu|)$,
  then $f_n$ and $f$ are $\nu$-integrable
  and $\int f_nd\nu\to\int fd\nu$.
  \item For $\nu$-integrable $f$, $|\int_\Omega f d\nu|\leq \int_\Omega |f|d|\nu|$.
  This is true for all indicators, hence true for all simple functions, then use density and dominated convergence.
\end{enumerate}

Let $\nu$ be a complex Borel measure on $K$. For $f\in C(K)$,
$\int_K |f|d|\nu|\leq ||f||_\infty |\nu|(K)<\infty$,
so $f$ is $\nu$-integrable.

Define $\phi:C(K)\to\mathbb{C}$, $\phi(f)=\int_K fd\nu$.
Then $\phi\in M(K)$ and $||\phi||\leq ||\nu||_1$.
If $\nu$ is a signed measure, then $\phi\in M^\mathbb{R}(K)$,
and if $\nu$ is a positive measure, then $\phi\in M^+(K)$.

\begin{theorem}[Riesz Representation Theorem]
For $\phi\in M^+(K)$,
there exists a unique positive Borel measure $\mu$ on $K$ that represents $\phi$:
$$\phi(f)=\int_K fd\mu\ \ \ \ \ \ \ \ \ \ \ \ \text{ for any $f\in C(K)$}.$$
\end{theorem}
\begin{proof}
  Uniqueness. Assume that $\mu_1$ and $\mu_2$ both represent $\phi$.
  Fix $E\subset U\subset K$, $E$ closed and $U$ open.
  By Urysohn's Lemma, there exists a function $f$
  such that $E\prec f\prec U$.
  Then
  $$\mu_1(E)\leq \int fd\mu_1=\int_K fd\mu_2\leq \mu_2(U).$$
  Taking the supremum over all $E$ closed and contained in $U$,
  we get $\mu_1(U)\leq \mu_2(U)$.
  Hence $\mu_1=\mu_2$ on $\mathcal{G}$ and hence (by regularity) on $\mathcal{B}$.

  Existence. We define an outer measure $\mu^K$ on $K$ as follows.
  For $U\in\mathcal{G}$ let $$\mu^*(U)=\sup\{\phi(f):f\prec U\}.$$
  Note that $0\leq \mu^*(U)\leq ||\phi||$.
  Also,
  $\mu^*(K)=\text{(as $\phi$ is positive)}\phi(1_K)=\text{(by Lemma 6)}||\phi||$.
  Also, if $U\subset V$ is open, then $\mu^*(U)\leq \mu^*(V)$.
  For $A\subset K$, $\mu^*(A)=\inf\{\mu^*(U):A\subset U\in\mathcal{G}\}$.
  This agrees with previous definition if $A\in\mathcal{G}$.

  Not finished.
\end{proof}


















%



\end{document}
