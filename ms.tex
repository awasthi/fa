\documentclass[11pt]{article}
% Mathematics
  % Packages
      \usepackage{amsmath, mathtools}
      \usepackage{amssymb}
      \usepackage{amsthm}
  % Theorem styling
    % Number within sections
      \newcounter{number_within_section}
      %\numberwithin{number_within_section}{section}
    % Theorem-style
      \newtheorem{theorem}    [number_within_section]{Theorem}
      \newtheorem*{theorem*}  {Theorem}
      \newtheorem{corollary}  [number_within_section]{Corollary}
      \newtheorem{lemma}      [number_within_section]{Lemma}
      \newtheorem{observation}[number_within_section]{Observation}
      \newtheorem{conjecture} [number_within_section]{Conjecture}
      \newtheorem{proposition}[number_within_section]{Proposition}
    % Definition-style
      \theoremstyle{definition}
      \newtheorem{definition} [number_within_section]{Definition}
      \newtheorem{example}    [number_within_section]{Example}
  % Create \numberthis command to number part of multi-line align* displays
      \newcommand\numberthis{\addtocounter{equation}{1}\tag{\theequation}}

% Spacing and typography
	% A4 lengths
    	\newlength{\pagewidth}
    	\setlength{\pagewidth}{210mm}
    	\newlength{\pageheight}
    	\setlength{\pageheight}{297mm}

	% Set page size and margins
		% 0.15625 = 10/64
  		\usepackage[a4paper,
  			left=   0.15625 \pagewidth,
  			right=  0.15625 \pagewidth,
  			top=    0.15625 \pageheight,
  			bottom= 0.15625 \pageheight
  		]{geometry}


% Section styling
    	\usepackage{titlesec}
    	\titleformat{\section}{\normalfont\scshape\large\filcenter}{Lecture \thesection, }{0pt}{}
    	\titlespacing*{\section}{0em}{2em}{1em}

% Times new roman
      \usepackage{mathptmx}

\begin{document}

\section*{\LARGE Functional analysis}

\leftskip=1.5cm\rightskip=1.5cm

This course in functional analysis was lectured by Andr\'as Zs\'ak at the University of Cambridge during Michaelmas term 2017,
and these notes have been produced by Piet Lammers.
Please feel free to use these notes as you see fit, but do so at your own risk.

\leftskip=0cm \rightskip=0cm


\section{6 October 2017.}
% !TEX root = ../ms.tex
We recall the definitions of a \emph{vector space} and of a \emph{(linear) functional} and of a \emph{(semi)norm} on a vector space.
A \emph{normed space} is a vector space equipped with a norm.
A \emph{Banach space} is a complete normed space over either $\mathbb{R}$ or $\mathbb{C}$.
The letter $X$ will usually denote a normed space,
and we write $B_X$ and $S_X$ for the unit ball and unit sphere of $X$ respectively.
Let $X$ be a normed space over some field $k\in\{\mathbb{R},\mathbb{C}\}$
and let $f:X\to k$ be a linear functional.
We call $f$ bounded if
\begin{equation}
  \sup_{x\in B_X}|f(x)|<\infty.
  \label{dualnorm}
\end{equation}
Note that $f$ is bounded if and only if it is continuous.
The collection of all bounded linear maps $f$ is called the \emph{dual space} of $X$,
and we write
$$X^*:=\{f:X\to k:\text{$f$ is linear and continuous}\}.$$
The left hand side of (\ref{dualnorm}) norms this vector space,
and is straightforward to check that this space is a Banach space.


Let $X$ and $Y$ be two normed spaces over $k\in\{\mathbb{R},\mathbb{C}\}$.
Write $X\sim Y$ if they there exists a vector space isomorphism that is continuous
and with a continuous inverse.
Write $X\cong Y$ if they are isometrically isomorphic, meaning that the isomorphism preserves
the norm.
Finally write $\langle x,f\rangle$ for $f(x)$, so that
$\langle\cdot,\cdot\rangle$ is a bilinear map from $X\times X^*$ to $k$.
It is straightforward to check that $|\langle x,f\rangle|\leq ||x||_X||f||_{X^*}$.

Now we state the Hahn-Banach theorem.
Let $X$ be a real vector space.
A functional $p:X\to \mathbb{R}$ is called
\emph{positive homogeneous} if $p(tx)=tp(x)$ for any $t\geq 0$, $x\in X$,
and \emph{subadditive} if $p(x+y)\leq p(x)+p(y)$ for any $x,y\in X$.
The functional $p$ is called sublinear if it satisfies both these conditions.
Norms, seminorms and linear functionals are examples of sublinear functionals.

\begin{theorem}[Hahn-Banach]\label{hb1}
  Let $X$ be a real vector space and $p$ a sublinear functional.
  Let $Y$ be a subspace of $X$ and $g$ a linear functional of $Y$
  with $g\leq p$ on $Y$.
  Then $g$ extends to a linear functional $f$ of $X$ with $f\leq p$.
\end{theorem}

For the proof, we recall Zorn's lemma (which is equivalent to the axiom of choice).
Let $(P,\leq)$ be a partially ordered set.
A subset $C\subset P$ is called a \emph{chain} if $\leq$ totally orders $C$.
An element $x\in P$ is called \emph{maximal} if no element in $P$ is strictly larger than $x$
(i.e., $x\leq y$ implies $y=x$).

\begin{theorem*}[Zorn's lemma]
  If $P$ is a nonempty partially ordered set such that every chain has an upper bound,
  then $P$ contains a maximal element.
\end{theorem*}

\begin{proof}[Proof of the Hahn-Banach theorem]
  Let $P$ be the set of pairs $(h,Z)$ such that $Y\subset Z\subset X$
  and such that $h$ is a linear functional on $Z$ extending $g$ to $Z$ satisfying $h\leq p$ on $Z$.
  The set $P$ is partially ordered by extension, and nonempty as it contains
  the element $(g,Y)$.
  For a chain $\{(h_i,Z_i)\}_{i\in I}$ it is easy to construct an upper bound,
  one takes $Z=\cup_{i\in I}Z_i$ and for $h$ one takes
  $h(x)=h_i(x)$, where $i$ is such that $x\in Z_i$ ($h$ is clearly independent of this choice).
  By construction, $h$ is linear and satisfies $h\leq p$ on $Z$.
  Thus, by Zorn's lemma, $P$ contains a maximal element, say $(h,Z)$.
  To demonstrate that $Z=X$ we argue by contradiction.
  Assume that there exists an element $x\in X\setminus Z$, and define $Z_x:=Z+\mathbb{R}x$.
  Define $$h_\alpha:Z+\mathbb{R}x\to\mathbb{R},\,z+\lambda x\mapsto h(z)+\lambda\alpha.$$
  To finish the proof it suffices to demonstrates that there exists an $\alpha\in\mathbb{R}$
  such that $h_\alpha\leq p$ on $Z_x$.
  Thus we need to find $\alpha$ such that for all $\lambda\neq0$ and for any $z$,
  \begin{align*}
    h_\alpha(z+\lambda x)&\leq  p(z+\lambda x)\\
    h(z)+\lambda\alpha&\leq p(z+\lambda x)\\
    \lambda\alpha&\leq p(z+\lambda x)-h(z)\\
    \lambda\alpha&\leq p(\lambda z'+\lambda x)-h(\lambda z'),
  \end{align*}
  writing $z'=z/\lambda$ to obtain the last inequality.
  Using positive homogeneity of $p$, we only need to look at $\lambda=\pm 1$.
  So, to prove existence of $\alpha$, it suffices to show that
  $$\sup_{z\in Z} h(z)-p(z-x) \leq \inf_{z\in Z} p(z+x)-h(z).$$
  But
  $$h(z_1)-p(z_1-x)\leq p(z_2+x)-h(z_2)$$
  since
  $$h(z_1)+h(z_2)=h(z_1+z_2)\leq p(z_1+z_2)\leq p(z_1-x)+p(z_2+x).$$
  This finishes the proof of the Hahn-Banach theorem.
\end{proof}



\section{9 October 2017.}
% !TEX root = ../ms.tex
\begin{theorem}[Hahn-Banach]\label{hb2}
  Let $X$ be a vector space over $k\in\{\mathbb{R},\mathbb{C}\}$, $p$ a seminorm on $X$,
  $Y\subset X$ a subspace, and $g$ a linear functional on $Y$ satisfying
  $|g|\leq p$ on $Y$.
  Then $g$ extends to a linear functional $f$ on $X$ that satisfies $|f|\leq p$ on $X$.
\end{theorem}

\begin{proof}
Let us first look at the real case.
Note that $g\leq |g|\leq p$ on $Y$,
so that we may apply Theorem \ref{hb1} to extend $g$ to a linear functional $f$ on $X$ satisfying
$f\leq p$. But then $-f\leq p$ as well, since $-f(x)=f(-x)\leq p(-x)=p(x)$.
Hence $|f|\leq p$.

Now the complex case.
Think of $X$ as a real space and let $g_\mathbb{R}=\operatorname{Re}(g)$.
Then $|g_\mathbb{R}|\leq |g|\leq p$ on $Y$.
By the real case $g_\mathbb{R}$ extends to a $\mathbb{R}$-linear functional $f_\mathbb{R}$ of $X$ satisfying
$|f_\mathbb{R}|\leq p$ on $X$.
Define
$f$ by $f(x)=f_\mathbb{R}(x)-if_\mathbb{R}(ix)$;
it is straightforward to check that this is a $\mathbb{C}$-linear functional on $X$.
Now $\operatorname{Re}(f)|_Y=f_{\mathbb{R}}|Y=g_{\mathbb{R}}$ so $f|_Y=g$ by uniqueness.
We need to show that $|f|\leq p$.
Pick $x\in X$ and pick $\lambda\in S^1\subset \mathbb{C}$ such that $|f(x)|=\lambda f(x)=f(\lambda x)$.
Then $|f(x)|=f(\lambda x)=f_\mathbb{R}(\lambda x)\leq p(\lambda x)=p(x)$.
\end{proof}

If $X$ is a complex vector space then let $X_\mathbb{R}$ be $X$ as real vector space.
We have just proved that if $X$ is a complex normed vector space,
the map
$$(X^*)_\mathbb{R}\to(X_\mathbb{R})^*,\,f\mapsto Re(f)$$
is an isometric isomorphism.

\begin{corollary}
  Let $X$ be a normed space over $k\in\{\mathbb{R},\mathbb{C}\}$, $p$ a seminorm,
  and $x_0\in X$. Then there exists a linear functional $f$ such that $f(x_0)=p(x_0)$
  and $|f|\leq p$ on $X$.
\end{corollary}
(It is not clear to me what is the reason for including this corollary at this point)

\begin{theorem}[Hahn-Banach]\label{hb4}
  Let $X$ be a normed space over $k\in\{\mathbb{R},\mathbb{C}\}$.
  Then
  \begin{enumerate}
    \item For any $Y\subset X$ and $g\in Y^*$, $g$ extends to a linear functional
    on $X$ of equal operator norm,
    \item For any $x_0\in X$, there exists a linear functional $f$ of $X$
    of norm one that maps $x_0$ to $||x_0||$.
  \end{enumerate}
\end{theorem}

\begin{proof}
  For Statement 1, apply Theorem \ref{hb2} with $p(x)=||g||\cdot ||x||$.
  For Statement 2, apply Theorem \ref{hb2} with $p=||\cdot||$,
  and set $g:kx_0\to k,\, \lambda x_0\mapsto \lambda ||x_0||$,
  which is a linear functional of norm one (the case $x_0=0$ is trivial).
\end{proof}

The previous theorem may be viewed as a linear version of the Tietze extension theorem.
We remark that $X^*$ separates points of $X$: if $x$ and $y$ are distinct points of $X$
then there exists a bounded linear functional $f$ such that $f(x)\neq f(y)$.
The linear functional $f$ in Statement 2 of Theorem \ref{hb4} is called a \emph{norming functional}
 for $x_0$:
for $g\in B_{X^*}$,
$|\langle x_0,g\rangle|=|g(x_0)|\leq ||g||\cdot ||x_0||\leq ||x_0||$
so $||x_0||=\max\{|\langle x_0,g\rangle|:g\in B_{X^*}\}$,
and this maximum is achieved at $f$.
If $||x_0||=1$ then
the same linear functional is also called a \emph{support functional},
because $\{f=1\}$ defines a plane tangent to the unit ball in $X$ at $x_0$.


Let $X$ be a normed space. The \emph{bidual} or \emph{second dual}
of $X$ is $X^{**}=(X^*)^*$.
This is a Banach space with the operator norm.
For $\phi\in X^{**}$, $||\phi||_{X^{**}}:=\sup\{|\langle f,\phi\rangle|:f\in B_{X^*}\}$.
For $x\in X$ define $\hat x$ to be a linear functional on $X^*$ by $\hat x(f)=f(x)$.
In bracket notation, $\langle f,\hat x\rangle=\langle x,f\rangle$.
The map $\hat x$ is the evaluation map of functionals in $X^*$ at $x$.
It is clear that $\hat x$ is a linear map. Furthermore
$|\hat x(f)|=|f(x)|\leq ||f||\cdot ||x||$.
So $\hat x\in X^{**}$ and $||\hat x||\leq ||x||$.
In fact, from the existence of norming functionals (Theorem \ref{hb4}, Statement 2) it follows that
$||\hat x||= ||x||$. This proves the following theorem.

\begin{theorem}
  The map $\hat \cdot: X\to X^{**}$
  is an isometric ismorphism from $X$ to $\hat X\subset X^{**}$.
\end{theorem}
The map $\hat\cdot$ is called the \emph{canonical embedding}.
Note that $\hat X$ is closed in $X^{**}$ if and only if $X$ is complete.
Furthermore, the closure of $\hat X$ in $X^{**}$ is a Banach space,
containing an isometric copy of $X$. We have thus implicitly proved that the space $X$
has a completion.
We say that $X$ is \emph{reflexive} if $\hat X=X^{**}$, i.e.,
if the canonical embedding is surjective.
Note that it is possible $X$ is not reflexive even if
 $X$ and $X^{**}$ are isometrically isomorphic.
Examples of reflexive spaces include $L^p$ spaces (with $1<p<\infty$),
and examples of non-reflexive spaces include $c_0$, $l^1$ and $l^\infty$.











%



\end{document}
